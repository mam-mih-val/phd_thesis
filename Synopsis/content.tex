\subsection*{Общая характеристика работы}

\newcommand{\actuality}{\underline{\textbf{Актуальность темы.}}}
\newcommand{\aim}{\underline{\textbf{Целью}}}
\newcommand{\tasks}{\underline{\textbf{задачи}}}
\newcommand{\defpositions}{\underline{\textbf{Основные положения, выносимые на~защиту:}}}
\newcommand{\novelty}{\underline{\textbf{Научная новизна:}}}
\newcommand{\influence}{\underline{\textbf{Практическая значимость}}}
\newcommand{\reliability}{\underline{\textbf{Достоверность}}}
\newcommand{\probation}{\underline{\textbf{Апробация работы.}}}
\newcommand{\contribution}{\underline{\textbf{Личный вклад.}}}
\newcommand{\publications}{\underline{\textbf{Публикации.}}}

% \usepackage{lineno}
% \begin{linenumbers}
{\actuality} Уравнение состояния (EOS) - описывает фундаментальные свойства ядерной материи, ее  макроскопические свойства, обусловленные лежащими в основе сильными взаимодействиями.  
Вблизи плотности насыщения ядерной материи $\rho_{0}$, $\rho_{0}=0.16 фм^{-3}$,  EOS контролирует структуру ядер через энергию связи и несжимаемость $K_{nm}$~\cite[B][]{Danielewicz:2002pu}.
EOS также определяет толщину нейтронной оболочки в нейтронно-избыточных ядрах, свойства ядерной материи при экстремальных плотностях и/или температурах.
Предполагается, что такие условия достигаются в экспериментах по столкновению релятивистских тяжелых ядер или в нейтронных звездах и слияниях нейтронных звезд. 
Более того, исследования показывают, что столкновения тяжелых ионов при энергиях пучка  $E_{kin}$=1.23--10$A$~ГэВ (соответствующих энергиям в системе центра масс $\sqrt{s_{NN}}$ = 2.4--5~ГэВ)  и слияния нейтронных звезд обнаруживают сходные температуры (T $\sim$  50--100~МэВ ) и плотности барионов $\rho \sim (2-5)\rho_{0}$~\cite{Bzdak:2019pkr,Xu:2022mqn}.
Не ограничиваясь описанием свойств материи, состоящей только из протонов и нейтронов, EOS может также отражать появление новых степеней свободы, например, странных частиц в ядрах нейтронных звезд или кварков и глюонов в ультрарелятивистских столкновениях тяжелых ионов. 
Считается, что столкновения ультра-релятивистских тяжелых ионов на Большом адронном коллайдере (LHC) и релятивистском коллайдере тяжелых ионов (RHIC), где  плотность барионов крайне мала, привели к образованию новой формы материи с партонными степенями свободы, обычно называемой сильносвязанной кварк-глюонной материей (КГМ)~\cite{Esumi:2022uas}.
После открытия КГМ на коллайдере RHIC в 2005 году изучение уравнения состояния  квантовой хромодинамики (КХД) в области высоких барионных плотностей стали главной целью программ сканирования по энергии в экспериментах: STAR на коллайдере RHIC ($\sqrt{s_{NN}}$ = 3 - 27~ГэВ), NA61/SHINE на ускорителе  SPS ($\sqrt{s_{NN}}$ = 5.2 - 17~ГэВ)~\cite{NA61:2014lfx}, BM@N на ускорителе Nuclotron ($\sqrt{s_{NN}}$ = 2.3 - 3.5~ГэВ)~\cite{Senger:2022bzm} и   HADES на ускорителе  SIS18 ($\sqrt{s_{NN}}$ = 2.3 - 2.55~ГэВ)~\cite{HADES:2009aat}. 
Строящиеся ускорители FAIR ($\sqrt{s_{NN}}=3-5$~ГэВ) и NICA ($\sqrt{s_{NN}}=4-11$~ГэВ)  позволят изучить  область высоких барионных плотностей еще более детально.

Ключевую роль в открытии  КГМ  и  определении ее ключевых транспортных свойств сыграли измерения анизотропных коллективных потоков рожденных адронов. 
Величина анизотропных потоков определяется коэффициентами ряда Фурье $v_n$ в разложении азимутального распределения частиц относительно угла плоскости реакции $\Psi_{RP}$, определяемой осью пучка и вектором прицельного параметра~\cite{Voloshin:2008dg}:
\begin{eqnarray}
   \frac{dN}{d\varphi} \propto 1 + 2 \sum_{n=1} v_n \cos\left( n \left( \varphi - \Psi_{RP} \right) \right),
\end{eqnarray}
%
где $n$ -- порядок гармоники и  $\varphi$ -- азимутальный угол импульса частиц.
Коэффициенты потоков  $v_n$ определяются как средние косинусы разности углов  $\left( \varphi - \Psi_{RP} \right)$ по частицам и событиям: $v_n=\left\langle \cos\left( n \left( \varphi - \Psi_{RP} \right) \right) \right\rangle$.
Благодаря своей чувствительности к деталям начального состояния сильновзаимодействующей материи и ранним временам столкновения, первые два коэффициента разложения Фурье $v_1$ (направленный поток) и $v_2$ (эллиптический поток) зависят от EOS созданной материи.
Основополагающее ограничение на значения несжимаемости $K_{nm}$ ядерной материи  в диапазоне плотностей (2-5)$\rho_{0}$ было получено путем сравнения измерений направленного ($v_1$) и эллиптического ($v_2$)  потоков  протонов в Au+Au столкновениях при энергиях  $E_{kin}$ = 2--8$A$~ГэВ ($\sqrt{s_{NN}}$ = 2.7--4.3 ГэВ), выполненных экспериментом E895 на ускорителе AGS, c теоретическими предсказаниями~\cite{E895:1999ldn,E895:2000maf,E895:2001axb}. 
Однако, интерпретация данных направленного потока $v_1$ протонов требует включения в модель ``мягкого'' EOS с коэффициентом несжимаемости $K_{nm} \sim 210$ МэВ. 
Значения  для эллиптического потока $v_2$ лучше согласуются с более ``жестким'' уравнением состояния $K_{nm} \sim 380$ МэВ~\cite{Danielewicz:2002pu}. 
В дополнение, новые экспериментальные измерения первых двух гармоник коллективных потоков протонов, выполненные экспериментом  STAR на коллайдере RHIC для данных энергий, не согласуются с результатами эксперимента E895. Одна из возможных причин различия в результатах измерений может заключаться в том, что стандартный метод плоскости событий для измерений потоков, использовавшийся 15–20 лет назад экспериментом E895, не учитывал влияние непотоковых корреляций  на измерения $v_n$. 
К непотоковым корреляциям можно отнести следующие эффекты: адронные резонансы и вклад вторичных частиц, сохранение полного(поперечного) импульса, фемтоскопические корреляции. 
Высокоточные измерения направленного и эллиптического потоков в этой области энергий современными методами анализа подавляющими вклад непотоковых корреляций важны для дальнейшего ограничения значения EOS симметричной сильно-взаимодействующей материи.
В 2019 году эксперимент HADES (High Acceptance Di-Electron Spectrometer)~\cite{HADES:2009aat}, расположенный на ускорителе SIS-18 в GSI, набрал порядка 2 млрд событий столкновений Ag+Ag при энергиях $E_{kin}$ = 1.23 AГэВ и 1.58 AГэВ ($\sqrt{s_{NN}}$ = 2.4 ГэВ и 2.55 ГэВ), которые дополнили существующие данные для столкновений Au+Au при энергии  $E_{kin}$ = 1.23 AГэВ. Это позволило впервые провести высокоточные измерения направленного потока $v_1$ протонов используя современные методики подавляющие вклад непотоковых корреляций.
Ожидается, что сравнение результатов измерений для различных сталкивающихся систем при различных энергиях поможет  оценить вклад взаимодействия рожденных частиц с нуклонами-спектаторами в наблюдаемые коллективные  потоки и получить новые ограничения на значения EOS симметричной материи.
В феврале 2023 года закончился набор данных на первом в России эксперименте по изучению столкновений релятивистских ядер BM@N (Барионная Материя на Нуклотроне) на новом ускорительном комплексе NICA (ОИЯИ, Дубна), в ходе которого было набрано порядка 500 М событий столкновений ядер Xe+Cs(I) при энергии  $E_{kin}$ = 3.8 АГэВ. Данная работа впервые показала возможности измерения коллективных потоков
в эксперименте BM@N,  что значительно расширило его физическую программу по изучению EOS материи  в области высоких барионных плотностей.

\aim\ работы является экспериментальное исследование коллективной анизотропии протонов в ядро-ядерных столкновениях  Au$+$Au и Ag$+$Ag при энергиях $E_{kin}$=1.23--1.58$A$~ГэВ ($\sqrt{s_{NN}}$=2.4--2.55 ГэВ) в эксперименте HADES (GSI), а также  изучение возможности проведения измерений коллективной анизотропии в эксперименте BM@N (NICA, ОИЯИ).
Для~достижения поставленной цели необходимо решить следующие {\tasks}:
\begin{enumerate}
    \item  Усовершенствовать  и применить на практике  метод измерения коллективных потоков в экспериментах с фиксированной мишенью с учетом неоднородности азимутального аксептанса установки.

    \item  Разработать метод учета корреляций не связанных с коллективным движением рожденных частиц (непотоковых корреляций) и изучить их влияние на результаты измерения коллективных потоков.

    \item  Исследовать характеристики  направленного потока $v_1$ протонов в столкновениях Au$+$Au и Ag$+$Ag при энергиях $E_{kin}$=1.23--1.58$A$~ГэВ ($\sqrt{s_{NN}}$=2.4--2.55 ГэВ) в эксперименте HADES.

    \item  Произвести  сравнение полученных результатов измерения $v_1$ протонов с теоретическими моделями и данными  других экспериментов.

    \item  Исследовать влияние спектаторов налетающего ядра на формирование $v_1$ протонов с помощью проверки законов масштабирования коллективных потоков с энергией и геометрией столкновения.

    \item  Изучить возможности измерения  коллективных потоков протонов в эксперименте BM@N.
\end{enumerate}
\defpositions
\begin{enumerate}
    \item Зависимости коэффициента направленного потока $v_1$  протонов от центральности столкновения, поперечного импульса ($p_T$) и быстроты  ($y_{cm}$) для 
    столкновений Au$+$Au и Ag$+$Ag при энергиях $E_{kin}$ =1.23--1.58$A$~ГэВ ($\sqrt{s_{NN}}$=2.4--2.55 ГэВ) в эксперименте HADES.
    
    \item Метод учета вклада непотоковых корреляций и изучения их влияния на измеренные значения  коэффициентов потоков $v_n$ для экспериментов с фиксированной мишенью в условиях сильной неоднородности азимутального акспетанса установки.

    \item Результаты сравнения измеренных значений направленного потока ($v_1$) c расчетами в рамках современных моделей ядро-ядерных столкновений, проверка эффекта масштабирования  $v_1$ с энергией столкновения и геометрией области перекрытия.

    \item Получена оценка эффективности измерения коллективных потоков на экспериментальной установке BM@N.
\end{enumerate}
\novelty
\begin{enumerate}
    \item Впервые для экспериментов на фиксированной мишени разработаны и апробированы методы коррекции результатов измерения направленного потока на азимутальную неоднородность аксептанса установки и учета корреляций не связанных с  коллективным движением рожденных частиц.

    \item Впервые получены новые экспериментальные измерения направленного потока $v_1$ протонов с учетом вклада непотоковых корреляций для для ядро-ядерных столкновений (Au + Au, Ag + Ag) при энергиях $E_{kin}$ = 1.23-1.58$A$~ГэВ ($\sqrt{s_{NN}}$=2.4-2.55 ГэВ), позволяющие оценить вклад нуклонов-спектаторов в коллективную анизотропию частиц.
\end{enumerate}
\influence\ данной работы заключается в том, что новые прецизионные результаты измерения направленного потока $v_1$ протонов современными методами анализа позволяющими оценить вклад непотоковых корреляций являются принципиально важными для проверки и дальнейшего развития теоретических моделей ядро-ядерных столкновений, получению новых ограничений на значения EOS симметричной сильно-взаимодействующей материи в области максимальной барионной плотности.
Методика измерения коллективных анизотропных потоков, опробованная впервые в эксперименте HADES (ГСИ, Дармштадт), была адаптирована к условиям установки BM@N (NICA, ОИЯИ), и усовершенствована с целью уменьшения систематической ошибки измерения. Методика была апробирована на основе моделирования детектора BM@N и анализа первых физических данных эксперимента по изучению Xe+Cs(I) столкновений при энергии $E_{kin}$ = 3.8 АГэВ.  Данные результаты важны и  для будущего эксперимента MPD (NICA), 
который также может работать в моде эксперимента на фиксированной мишени.

\reliability\ полученных результатов подтверждается их согласованностью с опубликованными данными для измерения  $v_1$ протонов в столкновениях Au + Au при энергии 1.23$A$~ГэВ. Результаты измерения для наклона направленного потока $dv_1/dy|_{y=0}$ в области средних быстрот находятся в хорошем согласии со значениями с других экспериментов (STAR, FOPI)  и следуют зависимости от энергии столкновения и законам масштабирования коллективных потоков в данной области энергий.
Зависимости направленного потока ($v_1$ ) протонов от быстроты и поперечного импульса также согласуются с расчетами Монте-Карло моделей со импульсно-зависимым потенциалом~\cite{nara2019jam}, такими как JAM и UrQMD.
Для разработанных методов измерения коллективных анизотропных потоков была была исследована эффективность их измерений в эксперименте BM@N с помощью Монте-Карло моделирования и последующий полной реконструкции событий.
Хорошее согласие между величинами  $v_n$,  полученными из анализа полностью реконструированных в BM@N частиц  и модельных данных, говорит о высокой эффективности установки для измерения коллективных потоков.\\

\probation\
Основные результаты работы докладывались~на российских и международных конференциях: 
Международная конференция «Ядро» (2020, 2021, 2024, Россия), 
Международный Семинар «Исследования возможностей физических установок на FAIR и NICA» (2021, Россия), 
Международная научная конференция молодых учёных и специалистов «AYSS» (2022, 2023, ОИЯИ), 
Международная конференция по физике элементарных частиц и астрофизике «ICPPA» (2020, 2022, Россия), 
Ломоносовская конференция по физике элементарных частиц (2023, Россия), 
XXV Международный Балдинский семинар по проблемам физики высоких энергий (2023, ОИЯИ), 
Международный Семинар «NICA» (2022, 2023, Россия).

\contribution\ Диссертация основана на работах, выполненных автором в рамках международных коллабораций: HADES (GSI) в 2019-2022~гг и BM@N (ОИЯИ) в 2022-2024~гг. 
Из работ, выполненных в соавторстве, в диссертацию включены результаты, полученные лично автором или при его определяющем участии в постановке задач, разработке методов их решения, анализе данных, а также в подготовке результатов измерений для публикации от лица коллабораций HADES и BM@N.
Кроме того, диссертант принимал участие в наборе экспериментальных данных и контроле их качества.

% \publications\ Основные результаты по теме диссертации изложены в 5 печатных работаx, которые опубликованы в переодических научных журналах, входящих в базы данных Web of Science и Scopus.
\ifthenelse{\equal{\thebibliosel}{0}}{% Встроенная реализация с загрузкой файла через движок bibtex8
    \publications\ 
}{% Реализация пакетом biblatex через движок biber
%Сделана отдельная секция, чтобы не отображались в списке цитированных материалов
    \begin{refsection}%
        % \printbibliography[heading=countauthornotvak, env=countauthornotvak, keyword=biblioauthornotvak, section=1]%
        % \printbibliography[heading=countauthorvak, env=countauthorvak, keyword=biblioauthorvak, section=1]%
        % \printbibliography[heading=countauthorconf, env=countauthorconf, keyword=biblioauthorconf, section=1]%
        \printbibliography[heading=countauthor, env=countauthor, keyword=biblioauthor, section=1]%
        \publications\ Основные результаты по теме диссертации изложены в \arabic{citeauthor} статьтях \nocite{Mamaev:2020lpi,Mamaev:2020qom,Mamaev:2023fpr,Mamaev:2023yhz,HADES:2020lob,Mamaev:2024,Mamaev:2024a,Mamaev:2024-1,Mamaev:2024-2}, которые опубликованы в переодических научных журналах, входящих в базы данных Web of Science и Scopus.
        % \arabic{citeauthorvak} из которых изданы в журналах, рекомендованных ВАК\nocite{vakbib1,vakbib2}, 
        % \arabic{citeauthorconf} "--- в тезисах докладов\nocite{confbib1,confbib2}.
    \end{refsection}
}

% \end{linenumbers}
 % Характеристика работы по структуре во введении и в автореферате не отличается (ГОСТ Р 7.0.11, пункты 5.3.1 и 9.2.1), потому её загружаем из одного и того же внешнего файла, предварительно задав форму выделения некоторым параметрам

%Диссертационная работа была выполнена при поддержке грантов ...

%\underline{\textbf{Объем и структура работы.}} Диссертация состоит из~введения, четырех глав, заключения и~приложения. Полный объем диссертации \textbf{ХХХ}~страниц текста с~\textbf{ХХ}~рисунками и~5~таблицами. Список литературы содержит \textbf{ХХX}~наименование.

% \begin{linenumbers}
%\newpage
\subsection*{Содержание работы}
Во \underline{\textbf{введении}} обосновывается актуальность исследований, проводимых в рамках данной диссертационной работы, приводится обзор научной литературы по изучаемой проблеме, формулируется цель, ставятся задачи работы, сформулированы научная новизна и практическая значимость представляемой работы.

В первом разделе \underline{\textbf{первой главы}}   кратко обсуждается статус теоретических и экспериментальных
исследований направленного ($v_1$) и эллиптического $v_2$ потоков, которые начались более 30 лет назад
в экспериментах на ускорителе BEVALAC, как схематически изображено на рис~\ref{fig:bounce_off}

\begin{figure}[h]
\begin{center}
  \includegraphics[width=0.55\linewidth]{images/flow1.png}
 \includegraphics[width=0.4\linewidth]{images/illustration_1_initial_spatial_asymmetry.png}
\caption{ Cхематичное изображение механизмов происхождения направленного $v_1$ и эллиптического $v_2$ потоков в
  столкновениях ядер при энергиях $\sqrt{s_{NN}}$=2-4 ГэВ (слева) и  при энергиях
  $\sqrt{s_{NN}}>$ 25 ГэВ (справа).}
\label{fig:bounce_off}
\end{center}
\end{figure}

При высоких энергиях столкновения $\sqrt{s_{NN}} > 25$ ГэВ, когда время прохождения сталкивающихся ядер
($t_{pass}<$ 1 фм/c), в гармониках $v_n$ потока доминирует гидродинамическое
коллективное расширение КГМ, вызванное наличием начальной пространственной анизотропии области перекрытия
ядер и геометрической флуктуации ее формы, которую можно охарактеризовать набором коэффициентов
эксцентриситета $\varepsilon_n$. Константа пропорциональности между $v_n$  и $\varepsilon_n$ оказывается чувствительной
к транспортным свойствам КГМ, таким как соотношения вязкости сдвига к плотности энтропии $\eta/s$.
Детальное сравнение модельных расчетов с измерениями $v_n$, что КГМ при энергиях
RHIC и LHC по своим свойствам является сильновзаимодействующей, близкой к идеальной, жидкости
со значением $\eta/s$, близким к постулированному минимуму 1/$(4\pi) \simeq$ 0.08.
Сравнение с расчетами транспортных моделей (RQMD, UrQMD и JAM), в которых нет
формирования КГМ, показывают, что перерассеяние адронов может воспроизвести только 20-30\%
от наблюдаемой величины сигнала  ($v_n$) потоков на RHIС.
В ходе программ сканирования по энергии на коллайдере RHIC от
3.0 до 200 ГэВ коллаборацией STAR было получено много интересных результатов для коллективных
потоков.
Наклон прямого потока $dv_1/dy$ в области средних быстрот ($y \sim 0$) для протонов и разницы
между наклоном $dv_1/dy$ у протонов и анти-протонов показывает сильно немонотонную зависимость от
энергии столкновения в области от 3.0 до 39 ГэВ. Это может указывать на “смягчение“
уравнения состояния в результате фазового перехода первого рода. Модельные расчеты показывают, что вклад
взаимодействия рожденных частиц со спектаторами становится значительным для энергий столкновения
менее чем $\sqrt{s_{NN}}\sim$7 ГэВ. При уменьшении энергии $v_2$ протонов меняет свой знак от положительного
(при энергии около 4 ГэВ) к отрицательному. В области энергий порядка $\sqrt{s_{NN}}\sim$ 2-5 ГэВ для описания
$v_1$ и $v_2$ протонов необходимо использовать   транспортные модели с импульсно-зависимым потенциалом среднего поля, такие как JAM.\\
Во втором разделе \underline{\textbf{первой главы}}  рассматриваются методы экспериментального измерения азимутальных потоков. 
Все методы измерения потоков можно
сформулировать в терминах двух основных величин:  единичного вектора частиц $u_{n,k} = ( \cos n\varphi_k, \sin n\varphi_k )$ в плоскости поперечной оси пучка и
вектора потока $Q_n$ гармоники $n$:
\begin{equation}
    Q_n = \frac{1}{C} \sum_{k=1}^{N} w_k u_{n,k} = \frac{|Q_n|}{C} (\cos{\Psi_n}, \sin{\Psi_n}) ,
\end{equation}
где $k$ --- индекс частицы в группе, $w_k$ --- вес $k$-го вектора, $N$ --- множественность частиц в группе, $\Psi_n$ --- угол плоскости симметрии данного события, $|Q_n|$ ---
модуль $Q_n$-вектора и $C$ --- нормировочный коэффициент.  Угол $\varphi_k$ является азимутальным углом вылета частицы $k$ или азимутальной координатой $k$-го
элемента сегментированного детектора. Для треков  вес $w_k$  может быть единицей или заданной функцией $p_T$. Для сегментированных детекторов в качестве
веса $w_k$ используется сигнал, наблюдаемый в $k$-м сегменте детектора: заряд частиц или энергия.
Сумма проходит по всем частицам  в случае трекового детектора  или по модулям детекторов с
азимутальной сегментацией. В методе плоскости события (EP) нормировочный коэффициент $C$ выбирается $C=|Q_n|$, а в методе  скалярного произведения (SP)
 производится на множественность частиц в $Q_n$:  $C=\sum_{k=1}^N w_k$. \\
 В области энергий 1-4 АГэВ направленный поток $v_1$ является доминирующем сигналом, который не меняет свой знак,  поэтому высшие гармоники вычисляются
 относительно  плоскости симметрии первой гармоники $\Psi_1$. 
 Направленный поток $v_1$ определяется как проекция $u_1$-вектора на плоскость симметрии события:
%
\begin{equation}
    v_1 =  \frac{ \langle u_1 Q_1 \rangle }{R_1},
    \label{eq:v1_formula}
\end{equation}
где $R_1$ - разрешение плоскости симметрии и угловые скобки обозначают среднее по всем частицам в событии и по всем событиям. 
Эллиптический поток $v_2$ относительно первой плоскости симметрии определятся как корреляция $u_1$-вектора   с двумя
$Q_1$-векторами:

\begin{equation}
    v_2 = \frac{\langle u_2 Q_1^a Q_1^b \rangle}{R_1\{a\} R_1\{b\}},
    \label{eq:v2_formula}
\end{equation}
%
где индексы $a$ и $b$ означают две плоскости симметрии вычисленные отдельно.
Для вычисления $R_1$  используется метод, называемый методом трёх подсобытий, который можно выразить формулой:
%
\begin{equation}
    R_1\{a(b,c)\}  =  \sqrt { \frac{ \langle Q_1^a Q_1^b \rangle \langle Q_1^a Q_1^c \rangle }{ \langle Q_1^b Q_1^c \rangle} },
\end{equation}
%
где $a$, $b$ и $c$ --- три различных группы частиц, в каждой из которых $Q_1$-вектор вычислялся независимо.
Сравнивая $R_1$, вычисленный с использованием различных комбинаций $Q_1$ (к примеру $R_1\{a(b,c)\}$ и $R_1\{a(b,d)\}$) можно оценить вклад непотоковых корреляций в полученный $R_1$.
Сравнивая $v_1$ полученный относительно различных плоскостей симметрии (к примеру, $v_1\{a\}$ и $v_1\{b\}$), можно оценить вклад непотоковых
корреляций в результаты измерения. \\
Неоднородность азимутального аксептанса детектора  искажает распределение $Q_1$-вектора, которое в идеальном случае должно быть равномерным. 
Для коррекции этого эффекта был использован метод, описаный в работе~\cite{Selyuzhenkov:2007zi}.
Поскольку плоскость реакции распределена равномерно, в пределах большого количества столкновений формулу~(\ref{eq:v1_formula}) можно преобразовать следующим образом:
%
\begin{equation}
    v_1 =  2\frac{ \langle x_1 X_1 \rangle }{R_1^X} = 2\frac{ \langle y_1 Y_1 \rangle }{R_1^Y},
    \label{eq:v1xy_formula}
\end{equation}
%
где $x_1$ и $y_1$ --- компоненты $u_1$-вектора, $X_1$ и $Y_1$ --- компоненты $Q_1$-вектора и $R_1^{X,Y}$ --- разрешение плоскости симметрии, вычисленное при
помощи корреляций компонент $Q_1$-векторов.
Систематический вклад азимутальной неоднородности аксептанса детектора может быть оценен сравнением результатов полученных с использованием различных компонент $u_1$ и $Q_1$-векторов. 



\underline{В первом разделе \textbf{второй главы}}  приведено описание эксперимента на фиксированной мишени  HADES (High Acceptance Di-Electron Spec-
trometer)  ~\cite{HADES:2009aat}  на ускорителе SIS-18 (GSI, Германия). Спектрометр HADES состоит из тороидального сверхпроводящего магнита,
центрированного вокруг оси пучка, с шестью одинаковыми секторами регистрации (см. рис.~\ref{fig:hades_bmn_layouts} (слева)).
Трекинговая система, состоит из четырёх плоскостей многопроволочных дрейфовых камер (MDC).
Для измерения времени пролёта, установка оборудована времяпролётными системами TOF и RPC.
Идентификация частиц проводится одновременно  методом времени-пролета и по энерговыделению в камерах MDC.
Сцинтилляционный годоскоп Forward Wall (FW), расположенный на расстоянии 6.8 m от мишени, предназначен для 
регисмтрации заряженных фрагментов спектаторов налетающего ядра.


%
\begin{figure}[h]
\begin{center}
\includegraphics[width=0.4\linewidth]{images/hades_layout.jpg}
%\includegraphics[width=0.35\linewidth]{images/FW_layout.jpg}
\includegraphics[width=0.55\linewidth]{images/BM@N_layout.png}
\caption{Схема экспериментов HADES (слева) и BM@N (справа).}
\label{fig:hades_bmn_layouts}
\end{center}
\end{figure}


В 2012 году эксперимент HADES набрал порядка 1 млрд событий  столкновения Au+Au при кинетической
энергии пучка $E_{kin}$ = 1.23 AГэВ ($\sqrt{s_{NN}}$ = 2.4 ГэВ). В 2019 году
эти измерения были дополнены данными: 500 млн столкновений Ag+Ag
при энергиях пучка $E_{kin}$ = 1.23 AГэВ и 1.58 AГэВ ($\sqrt{s_{NN}}$ = 2.4 ГэВ и 2.55 ГэВ). Анализ этих данных лег в основу данной работы.

\underline{Во втором разделе \textbf{второй главы}} представлено краткое описание экспериментальной установки BM@N (см. рис.~\ref{fig:hades_bmn_layouts} (справа)).
Центральная трековая система BM@N состоящая из
4 станций двусторонних кремниевых микрополосковых датчиков (STS) и 7 станций
камер газообразных электронных умножителей (GEM) расположена внутри дипольного магнита с большой апертурой,
что позволяет восстанавливать импульс $p$ заряженных частиц c разрешением  $\Delta p/p \sim$ 1.7-2.5$\%$ для
энергий порядка 4A~ГэВ (магнитное поле 0.8 T). Время-пролетная система, состоящая из двух детекторов TOF400 и TOF700, используется для идентификации заряженных частиц.
Три передних детектора:
передний адронный калориметр (FHCal), кварцевый годоскоп (Hodo) и  сцинтилляционная стенка
(ScWall) предоставляют информацию о фрагментах спектаторов налетающего ядра.
В феврале 2023 года закончился первый  набор физических 
данных, в ходе которого BM@N набрал порядка 500 М событий столкновений ядер Xe+Cs(I) при энергии  $E_{kin}$ = 3.8 АГэВ.



%


\underline{\textbf{Третья глава}} Расматривается разработанная автором методика измерения направленного потока $v_1$ протонов
в эксперименте HADES. Первая часть посвящена отбору событий для анализа. 
Для анализа использовались только события столкновений, вершина которых лежала в следующих границах: $\sqrt{x_v^2+y_v^2}<3$~мм и $z_v \in (-60, 0)$~мм.
 Треки частиц были выбраны в
соответствии с параметром качества $\chi^2<100$, предоставленным алгоритмом фитирования трека, и ограничением на минимальное расстояние сближения трека с вершиной (DCA): $(-10,10)$~mm. Протоны идентифицировались при помощи информации из системы  TOF+RPC (см. рис.~\ref{fig:hades_pid} (слева)).

\begin{figure}[h]
\begin{center}
  \includegraphics[width=0.55\linewidth]{images/hades_pid_plot.png}
  \includegraphics[width=0.35\linewidth]{images/hades_mult.png}
  \caption{ Слева: распределение частиц со скоростью $\beta$  в зависимости от  импульса частицы к заряду
    (p/q). Справа: пример распределения суммарной множественности хитов в системе TOF+RPC.}
\label{fig:hades_pid}
\end{center}
\end{figure}

Распределение суммарной множественности хитов в системе TOF+RPC было использовано для определения центральности столкновений, используя метод 
подгонки Монте-Карло версией модели Глаубера ~\cite{HADES:2017def}(см. рис.~\ref{fig:hades_pid} (справа)).






\begin{figure}[h]
\begin{center}
  \includegraphics[width=0.4\linewidth]{images/eta_pt_qvectors.png}
   \includegraphics[width=0.4\linewidth]{images/hades_phi_rapidity.png}
   \caption{ Схематическое изображение аксептанса $p_T$ vs $\eta$, использованного для построения пяти  $Q_1$ -векторов для анализа $v_1$ протонов (слева).
   Азимутальный аксептанс  протонов в плоскости $\phi$  vs $y_{cm}$ (справа). }
\label{fig:hades_qvectors}
\end{center}
\end{figure}
%
Для оценка плоскости симметрии модули детектора FW были разделены на 3 группы: центральные (W1), средние (W2) и периферические (W3). Это позволило определить три $Q_1$ -вектора (см. 
рис.~\ref{fig:hades_qvectors} (слева)). 
Для оценки систематической ошибки вызванной непотоковыми корреляциями были построены 2 дополнительных $Q_1$-вектора из треков протонов с 
$p_T < 2.0$~ГэВ и  $0.35 < y_{cm} < 0.55$ (Mf) и $-0.55 < y_{cm} < -0.35$ (Mb), см. рис~\ref{fig:hades_qvectors} (справа).\\
Рисунок~\ref{fig:hades_uq_corr} показывает сравнение $v_1^{uncor}$ протонов, полученного с использованием различных компонент $u_1$ и $Q_1$-векторов. 
После применения поправок на азимутальную анизотропию аксептанса, остаточный эффект составляет порядка 2\%.\\
%
\begin{figure}[h]
\begin{center}
\includegraphics[width=0.75\linewidth]{images/hades_u1W1_centrality.png}
\caption{Зависимость компонент корреляции $\langle u_1 Q_1 \rangle$ от центральности после применения поправок на азимутальную неоднородность детектора
 для  Au+Au и Ag+Ag столкновений.}
\label{fig:hades_uq_corr}
\end{center}
\end{figure}


Разрешение плоскости симметрии $R_1$, полученное с использованием различных комбинаций $Q_1$-векторов, показано на рис~\ref{fig:hades_w1_combinations}.
Разрешение $R_1\{W1(W2,W3)\}$ заметно отличается от значений, полученных при помощи других комбинаций. 
Этот эффект может быть объяснён наличием непотоковых корреляций между парами $Q_1$-векторов $W1$ и $W2$, $W2$ и $W3$.
Эти вектора не имеют значительного разделения по быстроте, поэтому в значительной степени могут содержать вклад непотоковых корреляций. 
В столкновениях Ag+Ag при обеих энергиях, $R_1\{W1(Mf,Mb)\}$ так же значительно отклоняется от среднего результата. 
Это может быть вызвано наличием корреляций из-за закона сохранения импульса между векторами $Mf$ и $Mb$. 
В столкновениях Au+Au этот эффект менее выражен в силу большей множественности рождённых частиц.
%
\begin{figure}[h]
\begin{center}
\includegraphics[width=0.75\linewidth]{images/W1_combinations.png}
\caption{Зависимость  разрешения $R_1$ от центральности для различных комбинаций $Q_1$-векторов
  для Au+Au и  Ag+Ag столкновений}
\label{fig:hades_w1_combinations}
\end{center}
\end{figure}


На рис.~\ref{fig:hades_w1w3} представлена зависимость  $v_1$ протонов от центральности  в столкновениях Au+Au при $E_{kin}=$1.23$A$~ГэВ,
измеренная при помощи различных комбинаций $Q_1$-векторов. 

%
\begin{figure}[h]
\begin{center}
\includegraphics[width=0.7\linewidth]{images/W1AndW3Nucleus.png}
\caption{Зависимость $v_1$ протонов от центральности  в столкновениях Au+Au при энергии $E_{kin}=$1.23$A$~ГэВ,
  измеренная при помощи различных комбинаций $Q_1$-векторов. Черной линией представлено среднее результатов полученных при помощи разделенных по быстроте комбинаций.}
\label{fig:hades_w1w3}
\end{center}
\end{figure}

Слева представлены значения $v_1$ измеренные относительно внутреннего подсобытия W1, справа --- внешнего подсобытия W3 и подсобытия из треков заряженных частиц Mf. 
Результаты для комбинаций подсобытий, разделенных по быстроте, таких как например, $W1(Mf,W3)$ и $W1(Mb,W3)$ согласуются между собой в пределах 2\%,
за исключением наиболее центральных событий. Значения $v_1$, измеренные относительно различных плоскостей симметрии $W1$ и $W3$, так же согласуется в пределах 2\%.
Результаты для $v_1$, полученные с использованием комбинаций не разделенных по быстроте $Q_1$-векторов (например $W1(W2,W3)$) значительно отличаются.
В дальнейшем в качестве значений $v_1$ протонов было использовано среднее по всем комбинациям $Q_1$-векторов, разделенных по быстроте.\\
В  \underline{\textbf{четвертой главе}} приведены основные 
результаты измерения направленного потока $v_1$ протонов в  эксперименте HADES и результаты исследования эффективности измерения коллективных потоков на
экспериментальной установке BM@N.\\
На рис.~\ref{fig:hades_prl} показана зависимость измеренного направленного потока  $v_1$ протонов от поперечного импульса $p_T$ (слева) и быстроты $y_{cm}$ для 20-30\%
центральных Au+Au столкновений при кинетической энергии пучка $E_{kin}=1.23A$~ГэВ ~\cite{HADES:2020lob}. Для сравнения также показаны значения $v_1$   для дейтронов и тритонов.

\begin{figure}[h]
\begin{center}
\includegraphics[width=0.75\linewidth]{images/HADES_prl.png}
\caption{Зависимость  $v_1$ протонов от  поперечного импульса  $p_T$ (слева) и  быстроты $y_{cm}$ (справа) в 20-30\% центральных Au+Au столкновений при
  энергии $E_{kin}=$1.23$A$~ГэВ. Фигура взята из ~\cite{HADES:2020lob}.}
\label{fig:hades_prl}
\end{center}
\end{figure}

На рисунке~\ref{fig:hades_v1_ycm_pT} представлена зависимость  $v_1$ протонов от быстроты  $y_{cm}$  (слева) и поперечного импульса $p_T$ (справа) для среднецентральных
 Au+Au  и  Ag+Ag  столкновений. 
 Типичными источниками погрешностей измерений $v_1$ и их характерными
относительными значениями являются:  погрешности в реконструкции треков и определении
импульса частиц (1-3\%) в зависимости от $p_T$ и быстроты; из­менения критериев отбора кандидатов в протоны по квадрату массы (1-3\%); 
остаточная неоднородность азимутального аксептанса установки оценивалась как разница между компонентами $XX$ и $YY$ корреляции векторов $u_1$ и $Q_1$ (1-2\%);
сравнение результов полученных методами плоскости события и скалярного произведения (1-4\%); вклад непотоковых корреляций, путем сравнения 
значений $v_1$, полученных относительно различных плоскостей симметрии (W1, W2, W3) и
деленных на поправочный коэффициент разрешения, рассчитанный с использованием различных комбинаций $Q_1$-векторов (3-5\%).\\
 Значения $v_1$ протонов, в столкновениях Au+Au и Ag+Ag при  энергии $E_{kin}=$1.23$A$~ГэВ, хорошо согласуются с учетом систематических ошибок. 
 Протоны, рожденные в столкновениях Ag+Ag при большей энергии $E_{kin}=$1.58$A$~ГэВ обладают меньшим значением $v_1$.
 Модель JAM~\cite{nara2019jam} с импульсно зависимым потенциалом среднего поля хорошо описываетз ависимость $v_1$ от быстроты $y_{cm}$, как показано линиями на
 рис~\ref{fig:hades_v1_ycm_pT}.
Однако модель не способна описать зависимость $v_1$ протонов от поперечного импульса $p_T$.
%
\begin{figure}[h]
\begin{center}
\includegraphics[width=0.4\linewidth]{images/v1_hades_ycm.png}
\includegraphics[width=0.4\linewidth]{images/v1_hades_pT.png}
\caption{Зависимость $v_1$ от  $y_{cm}$ (слева)  и $p_T$ (справа) для столкновений Au+Au при энергии $E_{kin}=$1.23$A$~ГэВ
  и Ag+Ag при энергиях $E_{kin}=$1.23$A$ и 1.58$A$~ГэВ. Линиями показаны данные, полученные из модели JAM.}
\label{fig:hades_v1_ycm_pT}
\end{center}
\end{figure}

Зависимость  $v_1$ протонов от  быстроты $y_{cm}$ была параметризована кубической функцией $v_1(y_{cm}) = a_0 + a_1 y_{cm} + a_3 y_{cm}^3$. 
Затем, наклон направленного потока  $dv_1/dy_{cm}|_{y_{cm}=0}$ в области средних быстрот $y_{cm}\sim0$ был извлечен как параметр $a_1$.
Полученные значения $dv_1/dy|_{y_{cm}=0}$ протонов в столкновениях \au{}  и \ag{} хорошо
согласуются с измерениями  других экспериментов ~\cite{FOPI:2011aa,STAR:2020dav} (см. рис. ~\ref{fig:hades_dv1_dy_sqrt_snn}).
%
\begin{figure}[h]
\begin{center}
\includegraphics[width=0.6\linewidth]{images/dv1_dy_sqrt_snn.png}
\caption{ 
  Зависимость  $dv_1/dy|_{y_{cm}=0}$ протонов в среднецентральных столкновениях от  энергии столкновения $\sqrt{s_{NN}}$.
}
\label{fig:hades_dv1_dy_sqrt_snn}
\end{center}
\end{figure}


На рис.~\ref{fig:hades_dv1dy_many_plot} слева приведена зависимость наклона $dv_1/dy_{cm}|_{y_{cm}=0}$ протонов от центральности столкновения.
Поскольку при большей энергии время пролета двух ионов $t_{pass}$ меньше, наклон направленного потока протонов в столкновениях Ag+Ag при энергии $E_{kin}=$1.58$A$~ГэВ 
заметно меньше, чем при $E_{kin}=$1.23$A$~ГэВ. 
Для коррекции на время $t_{pass}$, наклон  $dv_1/dy_{cm}|_{y_{cm}=0}$ был нормирован на быстроту пучка $y_{beam}$: $dv_1/dy'|_{y'=0}$, где $y'=y_{cm}/y_{beam}$.
За исключением наиболее центральных событий, зависимость наклона $dv_1/dy'|_{y'=0}$ от центральности описывается одной кривой для всех трех наборов данных,
см. центральную часть рис.~\ref{fig:hades_dv1dy_many_plot}.
В каждом классе центральности был вычислен средний прицельный параметр $\langle b \rangle$ из модели Глаубера.
Радиус ядра пропорционален корню кубическому из массового числа $r_N \propto A^{1/3}$.
Для учета зависимости от размера сталкиваемых ядер,  $\langle b \rangle$ в каждом классе по центральности был нормирован на $A^{1/3}$.
Наклон $dv_1/dy'|_{y'=0}$ как функция относительного прицельного параметра
$ \langle b \rangle / A^{1/3}$ представлен на рис~\ref{fig:hades_dv1dy_many_plot} справа.
Данное преобразование улучшило согласие зависимостей наклона в центральных событиях. 
%
\begin{figure}[h]
\begin{center}
\includegraphics[width=0.9\linewidth]{images/dv1dy_many_plot.png}
\caption{Зависимость $dv_1/dy|_{y=0}$ протонов от центральности: (слева) для $y=y_{cm}$, (в центре) для $y_{cm}$ номированной 
на быстроту пучка $y' = y_{cm}/y_{beam}$ и (справа) для $dv_1/dy'|_{y'=0}$ как функции $\langle b \rangle$ нормированного на  $A^{1/3}$.}
\label{fig:hades_dv1dy_many_plot}
\end{center}
\end{figure}


Во второй части \underline{\textbf{четвертой главы}} приведены результаты
детального изучения возможности измерения $v_1$ и $v_2$ протонов в столкновениях Xe+Cs(I)   в эксперименте  BM@N.
Эффективность  измерений была проверена  с помощью Монте-Карло моделирования   и последующий
полной реконструкции событий в среде BMNROOT. 
Для восстановления плоскости симметрии модули адронного калориметра FHCal были разделены на 3 группы согласно их псевдобыстроте (F1, F2 и F3), см. 
рис.~\ref{fig:bmn_subevents} слева.
Дополнительно для исследования вклада непотоковых корреляций были введены два $Q_1$-вектора из треков заряженных частиц. 
Вектор $Tp$ построен для протонов со значениями $0.4<y_{cm}<0.6$ и  $0.2<p_{T}<2.0$~$GeV/c$.
Вектор $T\pi$ формировался для отрицательных пионов с  $0.2<y_{cm}<0.8$ и $0.1<p_T<0.5 GeV/c$, соответственно.



\begin{figure}[h]
\begin{center}
\includegraphics[width=0.45\linewidth]{images/FHCal_layout.png}
\includegraphics[width=0.38\linewidth]{images/pT_ycm_protons.png}
\caption{
Слева: Схема разделения модулей FCHal по группам для определения плоскости симметрии события.
Справа: Кинематические окна для подсчета $Q_1$-векторов из треков заряженных частиц.
}
\label{fig:bmn_subevents}
\end{center}
\end{figure}



%
\begin{figure}[h]
  \begin{center}
  \includegraphics[width=0.45\linewidth]{images/bmn_phi_eta.png}
\includegraphics[width=0.35\linewidth]{images/v1_proton_correction_rapidity.png}
\caption{Азимутальный аксептанс трековой системы BM@N (слева). Сравнение  $v_1(y_{cm})$ протонов в  модельных столкновениях Xe+Cs 
  для различных компонент $u_1$-вектора (справа). Разными маркерами обозначены результаты до и после коррекции на азимутальную
  неоднородность аксептанса детектора. }
\label{fig:bmn_components}
\end{center}
\end{figure}

Азимутальный аксептанс трековой системы BM@N сильно неоднороден (см. рис.~\ref{fig:bmn_components} (слева)).
Результат  применения коррекций на азимутальную неоднородность представлен на рис.~\ref{fig:bmn_components} (справа).
Разными цветами обозначены значения $v_1(e_{cm})$ протонов, полученные с использованием различных компонент $u_1$-вектора. 
Открытые и закрытые маркеры обозначают результаты до и после коррекции, соответственно.
Черной линией обозначены значения $v_1$  извлеченные напрямую из модели JAM.
После применения 3 ступеней коррекции, результаты полученные при помощи $YY$ корреляции $u_1$ и $Q_1$-векторов, хорошо согласуются с результатами из модели.
Напротив, значения $v_1$, посчитанные с использованием $XX$-компонент, расходятся с модельными  $v_1$. 
Причиной может служить сильное отклонение частиц в направлении оси $x$ в магнитном поле. 
В связи с этим, в дальнейшем для анализа могут быть использованы лишь корреляции $YY$-компонент $u_1$ и $Q_1$-векторов.

На рис.~\ref{fig:bmn_combinations} представлена зависимость разрешения плоскостей симметрии F1, F2 и F3 от
центральности для различных комбинаций $Q_1$-векторов.
Значения $R_1$, полученные при помощи разделенных по быстроте комбинаций, согласуются между собой в пределах статистической ошибки для всех трех плоскостей симметрии. 
Значительное отличие значений $R_1$, полученных с использованием комбинаций не разделенных по быстроте $Q_1$-векторов, может быть объяснено распространением адронного ливня в поперечном направлении, что вызывает дополнительные корреляции между векторами $F1$ и $F2$, и $F1$ и $F3$.
%
\begin{figure}[h]
\begin{center}
\includegraphics[width=0.95\linewidth]{images/R1_F123_combinations_centrality.png}
\caption{Зависимость разрешения плоскостей симметрии F1-F3 от центральности для различных комбинаций $Q_1$-векторов, см. текст.}
\label{fig:bmn_combinations}
\end{center}
\end{figure}

На рис.~\ref{fig:bmn_v1_v2} слева представлен направленный поток протонов, как функция быстроты в Монте-Карло моделировании столкновений $Xe+Cs$ из модели JAM. 
На рис.~\ref{fig:bmn_v1_v2} справа показан эллиптический поток протонов, как функция поперечного импульса в Монте-Карло моделировании столкновений $Xe+Cs$ из модели JAM. 
Разными цветами обозначена разная энергия столкновений. 
Линии обозначают $v_1$ и $v_2$ извлеченные напрямую из модели без реконструкции. 
Маркерами обозначены результаты анализа Монте-Карло моделирования отклика детектора.
Между данными извлеченными из модели и результатами анализа после реалистичной цепочки реконструкции наблюдается согласие в пределах статистической ошибки. 
%
\begin{figure}[h]
\begin{center}
\includegraphics[width=0.45\linewidth]{images/v1_proton_tof_rapidity.png}
\includegraphics[width=0.45\linewidth]{images/v2_proton_tof_pT.png}
\caption{ 
 Зависимость $v_1$ протонов от  $y_{cm}$ (слева) и $v_2$ протонов от $p_{T}$ (справа) для Xe+Cs(I)
столкновений при энергиях 2 - 4 АГэВ. 
Показано сравнение результатов анализа полностью реконструированных в BM@N частиц (маркеры) и модельных данных (линии).}
\label{fig:bmn_v1_v2}
\end{center}
\end{figure}

В \underline{\textbf{заключении}} приведены основные результаты работы, которые заключаются в следующем:
%% Согласно ГОСТ Р 7.0.11-2011:
%% 5.3.3 В заключении диссертации излагают итоги выполненного исследования, рекомендации, перспективы дальнейшей разработки темы.
%% 9.2.3 В заключении автореферата диссертации излагают итоги данного исследования, рекомендации и перспективы дальнейшей разработки темы.
\begin{enumerate}
  \item Разработан метод учета корреляций не связанных с коллективным
движением рожденных частиц (непотоковых корреляций) и изучено
их влияние на результаты измерения коллективных потоков в области энергий 1.2-4 АГэВ. 
  \item Впервые получены зависимости  $v_1$ протонов от быстроты и поперечного импульса, а так же наклона $dv_1/dy_{cm}|_{y_{cm}}$ в 
  области средних быстрот в столкновениях \au{} при  энергии  $E_{kin}=$1.23$A$~ГэВ и \ag{} 
  при энергиях $E_{kin}=$1.23$A$ и $E_{kin}=$1.58$A$~ГэВ в эксперименте HADES. Полу­ченные новые результаты измерения $v_1$ протонов современными методами 
  анализа являются принципиально важными для проверки и
дальнейшего развития теоретических моделей ядро-ядерных столкновений.
  \item Обнаружено масштабирование направленного потока протонов с временем пролета ядер $t_{pass}$ и геометрией столкновения в области
   энергий $E_{kin}=$1.23$A$ и $E_{kin}=$1.58$A$~ГэВ, что позволяет оценить влияние спектаторов налетающего ядра на
формирование направленного потока протонов.
  \item На основе моделирования установки детально изучены возможности измерения коллективных потоков протонов на экспериментальной установке BM@N на ускорителе
NUCLOTRON-NICA (ОИЯИ, Дубна). Это позволо расширить существующую физическую программу эксперимента BM@N.
\end{enumerate}


%\newpage
% При использовании пакета \verb!biblatex! список публикаций автора по теме
% диссертации формируется в разделе <<\publications>>\ файла
% \verb!../common/characteristic.tex!  при помощи команды \verb!\nocite! 

\ifthenelse{\equal{\thebibliosel}{0}}{% Встроенная реализация с загрузкой файла через движок bibtex8
  \renewcommand{\refname}{\large \authorbibtitle}
  \nocite{*}
  \insertbiblioauthor                          % Подключаем Bib-базы
  %\insertbiblioother   % !!! bibtex не умеет работать с несколькими библиографиями !!!
}{% Реализация пакетом biblatex через движок biber
  \insertbiblioauthor                          % Подключаем Bib-базы
  \insertbiblioother
}

% \end{linenumbers}
