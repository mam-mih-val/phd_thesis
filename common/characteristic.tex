% \usepackage{lineno}
% \begin{linenumbers}
{\actuality} Уравнение состояния (EOS) - описывает фундаментальные свойства ядерной материи, ее  макроскопические свойства, обусловленные лежащими в основе сильными взаимодействиями.  
Вблизи плотности насыщения ядерной материи $\rho_{0}$, $\rho_{0}=0.16 фм^{-3}$,  EOS контролирует структуру ядер через энергию связи и несжимаемость $K_{nm}$~\cite[B][]{Danielewicz:2002pu}.
EOS также определяет толщину нейтронной оболочки в нейтронно-избыточных ядрах, а также свойства ядерной материи при экстремальных плотностях и/или температурах, соответствующих условиям, возникающим в экспериментах со столкновением релятивистских тяжелых ядер или наблюдаемым в нейтронных звездах  и слияниях нейтронных звезд. 
Более того, исследования показывают, что столкновения тяжелых ионов при энергиях пучка  $E_{kin}$=1.23--10$A$~ГэВ (соответствующих энергиям в системе центра масс $\sqrt{s_{NN}}$ = 2.4--5~ГэВ)  и слияния нейтронных звезд обнаруживают сходные температуры (T $\sim$  50--100~МэВ ) и плотности барионов $\rho \sim (2-5)\rho_{0}$~\cite{Bzdak:2019pkr,Xu:2022mqn}.
Не ограничиваясь описанием свойств материи, состоящей только из протонов и нейтронов, EOS может также отражать появление новых степеней свободы, например, странных частиц в ядрах нейтронных звезд или кварков и глюонов в ультрарелятивистских столкновениях тяжелых ионов. 
Считается, что столкновения ультра-релятивистских тяжелых ионов на Большом адронном коллайдере (LHC) и релятивистском коллайдере тяжелых ионов (RHIC), где  плотность барионов крайне мала, привели к образованию новой формы материи с партонными степенями свободы, обычно называемой сильносвязанной кварк-глюонной материей (КГМ)~\cite{Esumi:2022uas}.
После открытия КГМ на коллайдере RHIC в 2005 году изучение уравнения состояния  квантовой хромодинамики (КХД) в области высоких барионных плотностей стали главной целью программ сканирования по энергии в экспериментах: STAR на коллайдере RHIC ($\sqrt{s_{NN}}$ = 3 - 27~ГэВ), NA61/SHINE на ускорителе  SPS ($\sqrt{s_{NN}}$ = 5.2 - 17~ГэВ)~\cite{NA61:2014lfx}, BM@N на ускорителе Nuclotron ($\sqrt{s_{NN}}$ = 2.3 - 3.5~ГэВ)~\cite{Senger:2022bzm} и   HADES на ускорителе  SIS18 ($\sqrt{s_{NN}}$ = 2.3 - 2.55~ГэВ)~\cite{HADES:2009aat}. 
Строящиеся ускорители FAIR ($\sqrt{s_{NN}}=3-5$~ГэВ) и NICA ($\sqrt{s_{NN}}=4-11$~ГэВ)  позволят изучить  область высоких барионных плотностей еще более детально.

Ключевую роль в открытии  КГМ  и  определении ее ключевых транспортных свойств сыграли измерения анизотропных коллективных потоков рожденных адронов. 
Величина анизотропных потоков определяется коэффициентами ряда Фурье $v_n$ в разложении азимутального распределения частиц относительно угла плоскости реакции $\Psi_{RP}$, определяемой осью пучка и вектором прицельного параметра~\cite{Voloshin:2008dg}:
\begin{eqnarray}
   \frac{dN}{d\varphi} \propto 1 + 2 \sum_{n=1} v_n \cos\left( n \left( \varphi - \Psi_{RP} \right) \right),
\end{eqnarray}
%
где $n$ -- порядок гармоники и  $\varphi$ -- азимутальный угол импульса частиц.
Коэффициенты потоков  $v_n$ можно определить путем усреднения косинуса разности углов  $\left( \varphi - \Psi_{RP} \right)$ по частицам и событиям: $v_n=\left\langle \cos\left( n \left( \varphi - \Psi_{RP} \right) \right) \right\rangle$.
Благодаря своей чувствительности к деталям начального состояния сильновзаимодействующей материи и ранним временам столкновения, первые два коэффициента разложения Фурье $v_1$ (направленный поток) и $v_2$ (эллиптический поток) являются одними из самих чувствительных к EOS сигналами.
Основополагающее ограничение на значения несжимаемости $K_{nm}$ ядерной материи  в диапазоне плотностей (2-5)$\rho_{0}$ было получено путем сравнения измерений направленного ($v_1$) и эллиптического ($v_2$)  потоков  протонов в Au+Au столкновениях при энергиях  $E_{kin}$ = 2--8$A$~ГэВ ($\sqrt{s_{NN}}$ = 2.7--4.3 ГэВ), выполненных экспериментом E895 на ускорителе AGS, c теоретическими предсказаниями~\cite{E895:1999ldn,E895:2000maf,E895:2001axb}. 
Однако, интерпретация данных направленного потока $v_1$ протонов требует включения в модель ``мягкого'' EOS с коэффициентом несжимаемости $K_{nm} \sim 210$ МэВ. 
Значения  для эллиптического потока $v_2$ лучше согласуются с более ``жестким'' уравнением состояния $K_{nm} \sim 380$ МэВ~\cite{Danielewicz:2002pu}. 
В дополнение, новые экспериментальные измерения первых двух гармоник коллективных потоков протонов, выполненные экспериментом  STAR на коллайдере RHIC для данных энергий, не согласуются с результатами эксперимента E895. Одна из возможных причин различия в результатах измерений может заключаться в том, что стандартный метод плоскости событий для измерений потоков, использовавшийся 15–20 лет назад экспериментом E895, не учитывал влияние непотоковых корреляций  на измерения $v_n$. 
К непотоковым корреляциям можно отнести следующие эффекты: адронные резонансы и вклад вторичных частиц, сохранение полного(поперечного) импульса, фемтоскопические корреляции. 
Высокоточные измерения направленного и эллиптического потоков в этой области энергий современными методами анализа подавляющими вклад непотоковых корреляций важны для дальнейшего ограничения значения EOS симметричной сильно-взаимодействующей материи.
В 2019 году эксперимент HADES (High Acceptance Di-Electron Spectrometer)~\cite{HADES:2009aat}, расположенный на ускорителе SIS-18 в GSI, набрал порядка 2 млрд событий столкновений Ag+Ag при энергиях $E_{kin}$ = 1.23 AГэВ и 1.58 AГэВ ($\sqrt{s_{NN}}$ = 2.4 ГэВ и 2.55 ГэВ), которые дополнили существующие данные для столкновений Au+Au при энергии  $E_{kin}$ = 1.23 AГэВ. Это позволило впервые провести высокоточные измерения направленного потока $v_1$ протонов используя современные методики подавляющие вклад непотоковых корреляций.
Ожидается, что сравнение результатов измерений для различных сталкивающихся систем при различных энергиях поможет  оценить вклад взаимодействия рожденных частиц с нуклонами-спектаторами в наблюдаемые коллективные  потоки и получить новые ограничения на значения EOS симметричной материи.
В феврале 2023 года закончился набор данных на первом в России эксперименте по изучению столкновений релятивистских ядер BM@N (Барионная Материя на Нуклотроне) на новом ускорительном комплексе NICA (ОИЯИ, Дубна), в ходе которого было набрано порядка 500 М событий столкновений ядер Xe+Cs(I) при энергии  $E_{kin}$ = 3.8 АГэВ. Данная работа впервые показала возможности измерения коллективных потоков
в эксперименте BM@N,  что значительно расширило его физическую программу по изучению EOS материи  в области высоких барионных плотностей.

\aim\ работы является экспериментальное исследование коллективной анизотропии протонов в ядро-ядерных столкновениях  Au$+$Au и Ag$+$Ag при энергиях $E_{kin}$=1.23--1.58$A$~ГэВ ($\sqrt{s_{NN}}$=2.4--2.55 ГэВ) в эксперименте HADES (GSI), а также  изучение возможности проведения измерений коллективной анизотропии в эксперименте BM@N (NICA, ОИЯИ).
Для~достижения поставленной цели необходимо решить следующие {\tasks}:
\begin{enumerate}
    \item  Усовершенствовать  и применить на практике  метод измерения коллективных потоков в экспериментах с фиксированной мишенью с учетом неоднородности азимутального аксептанса установки.

    \item  Разработать метод учета корреляций не связанных с коллективным движением рожденных частиц (непотоковых корреляций) и изучить их влияние на результаты измерения коллективных потоков.

    \item  Исследовать характеристики  направленного потока $v_1$ протонов в столкновениях Au$+$Au и Ag$+$Ag при энергиях $E_{kin}$=1.23--1.58$A$~ГэВ ($\sqrt{s_{NN}}$=2.4--2.55 ГэВ) в эксперименте HADES.

    \item  Произвести  сравнение полученных результатов измерения $v_1$ протонов с теоретическими моделями и данными  других экспериментов.

    \item  Исследовать влияние спектаторов налетающего ядра на формирование $v_1$ протонов с помощью проверки законов масштабирования коллективных потоков с энергией и геометрией столкновения.

    \item  Изучить возможности измерения  коллективных потоков протонов в эксперименте BM@N.
\end{enumerate}
\defpositions
\begin{enumerate}
    \item Зависимости коэффициента направленного потока $v_1$  протонов от центральности столкновения, поперечного импульса ($p_T$) и быстроты  ($y_{cm}$) для 
    столкновений Au$+$Au и Ag$+$Ag при энергиях $E_{kin}$ =1.23--1.58$A$~ГэВ ($\sqrt{s_{NN}}$=2.4--2.55 ГэВ) в эксперименте HADES.
    
    \item Метод учета вклада непотоковых корреляций и изучения их влияния на измеренные значения  коэффициентов потоков $v_n$ для экспериментов с фиксированной мишенью в условиях сильной неоднородности азимутального акспетанса установки.

    \item Результаты сравнения измеренных значений направленного потока ($v_1$) c расчетами в рамках современных моделей ядро-ядерных столкновений, проверка эффекта масштабирования  $v_1$ с энергией столкновения и геометрией области перекрытия.

    \item Получена оценка эффективности измерения коллективных потоков на экспериментальной установке BM@N.
\end{enumerate}
\novelty
\begin{enumerate}
    \item Впервые для экспериментов на фиксированной мишени разработаны и апробированы методы коррекции результатов измерения направленного потока на азимутальную неоднородность аксептанса установки и учета корреляций не связанных с  коллективным движением рожденных частиц.

    \item Впервые получены новые экспериментальные измерения направленного потока $v_1$ протонов с учетом вклада непотоковых корреляций для для ядро-ядерных столкновений (Au + Au, Ag + Ag) при энергиях $E_{kin}$ = 1.23-1.58$A$~ГэВ ($\sqrt{s_{NN}}$=2.4-2.55 ГэВ), позволяющие оценить вклад нуклонов-спектаторов в коллективную анизотропию частиц.
\end{enumerate}
\influence\ данной работы заключается в том, что новые прецизионные результаты измерения направленного потока $v_1$ протонов современными методами анализа позволяющими оценить вклад непотоковых корреляций являются принципиально важными для проверки и дальнейшего развития теоретических моделей ядро-ядерных столкновений, получению новых ограничений на значения EOS симметричной сильно-взаимодействующей материи в области максимальной барионной плотности.
Методика измерения коллективных анизотропных потоков, опробованная впервые в эксперименте HADES (ГСИ, Дармштадт), была адаптирована к условиям установки BM@N (NICA, ОИЯИ), и усовершенствована с целью уменьшения систематической ошибки измерения. Методика была апробирована на основе моделирования детектора BM@N и анализа первых физических данных эксперимента по изучению Xe+Cs(I) столкновений при энергии $E_{kin}$ = 3.8 АГэВ.  Данные результаты важны и  для будущего эксперимента MPD (NICA), 
который также может работать в моде эксперимента на фиксированной мишени.

\reliability\ полученных результатов подтверждается их согласованностью с опубликованными данными для измерения  $v_1$ протонов в столкновениях Au + Au при энергии 1.23$A$~ГэВ. Результаты измерения для наклона направленного потока $dv_1/dy|_{y=0}$ в области средних быстрот находятся в хорошем согласии со значениями с других экспериментов (STAR, FOPI)  и следуют зависимости от энергии столкновения и законам масштабирования коллективных потоков в данной области энергий.
Зависимости направленного потока ($v_1$ ) протонов от быстроты и поперечного импульса также согласуются с расчетами Монте-Карло моделей со импульсно-зависимым потенциалом~\cite{nara2019jam}, такими как JAM и UrQMD.
Для разработанных методов измерения коллективных анизотропных потоков была была исследована эффективность их измерений в эксперименте BM@N с помощью Монте-Карло моделирования и последующий полной реконструкции событий.
Хорошее согласие между величинами  $v_n$,  полученными из анализа полностью реконструированных в BM@N частиц  и модельных данных, говорит о высокой эффективности установки для измерения коллективных потоков.\\

\probation\
Основные результаты работы докладывались~на российских и международных конференциях: 
Международная конференция «Ядро» (2020, 2021, 2024, Россия), 
Международный Семинар «Исследования возможностей физических установок на FAIR и NICA» (2021, Россия), 
Международная научная конференция молодых учёных и специалистов «AYSS» (2022, 2023, ОИЯИ), 
Международная конференция по физике элементарных частиц и астрофизике «ICPPA» (2020, 2022, Россия), 
Ломоносовская конференция по физике элементарных частиц (2023, Россия), 
XXV Международный Балдинский семинар по проблемам физики высоких энергий (2023, ОИЯИ), 
Международный Семинар «NICA» (2022, 2023, Россия).

\contribution\ Диссертация основана на работах, выполненных автором в рамках международных коллабораций: HADES (GSI) в 2019-2022~гг и BM@N (ОИЯИ) в 2022-2024~гг. 
Из работ, выполненных в соавторстве, в диссертацию включены результаты, полученные лично автором или при его определяющем участии в постановке задач, разработке методов их решения, анализе данных, а также в подготовке результатов измерений для публикации от лица коллабораций HADES и BM@N.
Кроме того, диссертант принимал участие в наборе экспериментальных данных и контроле их качества.

% \publications\ Основные результаты по теме диссертации изложены в 5 печатных работаx, которые опубликованы в переодических научных журналах, входящих в базы данных Web of Science и Scopus.
\ifthenelse{\equal{\thebibliosel}{0}}{% Встроенная реализация с загрузкой файла через движок bibtex8
    \publications\ 
}{% Реализация пакетом biblatex через движок biber
%Сделана отдельная секция, чтобы не отображались в списке цитированных материалов
    \begin{refsection}%
        % \printbibliography[heading=countauthornotvak, env=countauthornotvak, keyword=biblioauthornotvak, section=1]%
        % \printbibliography[heading=countauthorvak, env=countauthorvak, keyword=biblioauthorvak, section=1]%
        % \printbibliography[heading=countauthorconf, env=countauthorconf, keyword=biblioauthorconf, section=1]%
        \printbibliography[heading=countauthor, env=countauthor, keyword=biblioauthor, section=1]%
        \publications\ Основные результаты по теме диссертации изложены в \arabic{citeauthor} статьтях \nocite{Mamaev:2020lpi,Mamaev:2020qom,Mamaev:2023fpr,Mamaev:2023yhz,HADES:2020lob,Mamaev:2024,Mamaev:2024a}, которые опубликованы в переодических научных журналах, входящих в базы данных Web of Science и Scopus.
        % \arabic{citeauthorvak} из которых изданы в журналах, рекомендованных ВАК\nocite{vakbib1,vakbib2}, 
        % \arabic{citeauthorconf} "--- в тезисах докладов\nocite{confbib1,confbib2}.
    \end{refsection}
}

% \end{linenumbers}
