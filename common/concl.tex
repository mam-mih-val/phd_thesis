%% Согласно ГОСТ Р 7.0.11-2011:
%% 5.3.3 В заключении диссертации излагают итоги выполненного исследования, рекомендации, перспективы дальнейшей разработки темы.
%% 9.2.3 В заключении автореферата диссертации излагают итоги данного исследования, рекомендации и перспективы дальнейшей разработки темы.
\begin{enumerate}
  \item Разработан метод учета корреляций не связанных с коллективным
движением рожденных частиц (непотоковых корреляций) и изучено
их влияние на результаты измерения коллективных потоков в области энергий 1.2-4 АГэВ. 
  \item Впервые получены зависимости  $v_1$ протонов от быстроты и поперечного импульса, а так же наклона $dv_1/dy_{cm}|_{y_{cm}}$ в 
  области средних быстрот в столкновениях \au{} при  энергии  $E_{kin}=$1.23$A$~ГэВ и \ag{} 
  при энергиях $E_{kin}=$1.23$A$ и $E_{kin}=$1.58$A$~ГэВ в эксперименте HADES. Полу­ченные новые результаты измерения $v_1$ протонов современными методами 
  анализа являются принципиально важными для проверки и
дальнейшего развития теоретических моделей ядро-ядерных столкновений.
  \item Обнаружено масштабирование направленного потока протонов с временем пролета ядер $t_{pass}$ и геометрией столкновения в области
   энергий $E_{kin}=$1.23$A$ и $E_{kin}=$1.58$A$~ГэВ, что позволяет оценить влияние спектаторов налетающего ядра на
формирование направленного потока протонов.
  \item На основе моделирования установки детально изучены возможности измерения коллективных потоков протонов на экспериментальной установке BM@N на ускорителе
NUCLOTRON-NICA (ОИЯИ, Дубна). Это позволо расширить существующую физическую программу эксперимента BM@N.
\end{enumerate}
