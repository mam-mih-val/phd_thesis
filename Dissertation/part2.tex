\chapter{Экспериментальные установки HADES и BM@N} \label{chapt2}

\section{Эксперимент HADES}


HADES (High Acceptance DiElectron Spectrometer) является многофункциональной экспериментальной установкой на фиксированной мишени.
Установка базируется на отдельном выводе ускорителя SIS-18 в центре по изучению тяжелых ионов имени Гельмгольца ГСИ, в городе Дармштадт.
Физическая программа установки состоит из экспериментов по столкновению как адронов так и тяжелых ядер и направлена на изучение механизмов образования странных частиц и роли барионных резонансов в этих процессах. 
Схема экспериментальной установки приведена на рис.~\ref{fig:hades_bmn_layouts}~\cite{HADES:2009aat}.
Эксперимент состоит из 6 секторов, которые расположены радиально симметрично относительно оси пучка.
%
\begin{figure}[ht]
\begin{center}
\includegraphics[width=0.75\linewidth]{images/hades_layout.jpg}
\caption{Схема эксперимента HADES}
\label{fig:hades_bmn_layouts}
\end{center}
\end{figure}

\section{Ускорительный комплекс SIS-18}

Ускорительный комплекс SIS-18 состоит из линейного ускорителя UNILAC и синхротрона тяжелых ионов SIS-18.
Линейный ускоритель UNILAC способен разгонять ионы в широком диапазоне массовых чисел: от протонов до ядер урана.
Ускоритель оборудован инжектором ионов VARIS (Vacuum Arc Ion Source), способного достигать силы тока ионов до 6~мА.
При помощи вакуумно-дугового разряда тяжелые ионы испаряются с поверхности источника, а затем разделяются с помощью масс-спектрометра в подсистеме LEBT (Low Energy Beam Transport system).
Затем ионы тяжелых ядер с энергией 2.2$A$~кэВ транспортируются в инжектор High Current Injector, в котором они разгоняются до энергий 1.4$A$~МэВ и полностью лишаются электронной оболочки с помощью сверхзвукового потка газа.
В дальнейшем полностью ионизированные тяжелые ядра при энергии 11.4$A$~МэВ подаются на вход синхротнона SIS-18.

Максимальная магнитная жесткость синхротрона достигает 18~Тм, что позволяет разогнать ядра $Au^{69+}$ до 1.25$A$~ГэВ, $Ag^{47+}$ до 1.5$A$~ГэВ и протоны до 4.5$A$~ГэВ.
Длина синхротронного кольца составляет 217~м.
Кольцо разделено на 12 идентичных секций.
Каждая секция состоит из двух дипольных магнитов для отклонения пучка, трех квадрупольных магнитов и одного секступольного магнита для фокусировки пучка.
После синхротрона ускоренные тяжелые ядра подаются на вход эксперимента HADES.

\subsection{ Мишень }

Мишень, на которой происходит взаимодействия ускоренных ядер представляет из себя 15 каптоновых полосок, закрепленных на углеволоконной трубке.
На каптоновые полоски толщиной 7~мкм наклеены диски из золота (серебра) толщиной 25~мкм.
Расстояние между полосками составляет 4 мм. 
Общая толщина мишени --- 375~мкм, что соответствует общей вероятности взаимодействия в 1.5\%.
Фотография мишени приведена на рис.~\ref{fig:hades_target}. 
%
\begin{figure}[ht]
\begin{center}
\includegraphics[width=0.75\linewidth]{images/hades_target.jpg}
\caption{ Фотография мишени для  }
\label{fig:hades_target}
\end{center}
\end{figure}

\subsection{ Магнитный спектрометр }

Магнитный спектрометр состоит из 6 тороидальных сверхпроводящих магнитов и 24 многопроволочных камер (MDC).
В каждом из 6 секторов две плоскости MDC располагаются до магнита и 2 --- после. 
Реконструкция импульса производится посредством итеративного решения уравнения движения частиц в локальном магнитном поле при помощи метода Рунге-Кутта.
Заряд частицы определяется по отклонению частицы в магнитном поле: положительно заряженные частиц отклоняются в сторону оси Z, отрицательно заряженные --- против.
Реконструкция импульса может выполняться для большого окна полярных углов в широком диапазоне $p$: от 0.1 до 2 ГэВ/c.
Магнитное поле позволяет измерять импульсы заряженных частиц с разрешением для электронов с энергией 0.15~МэВ $\Delta p/p \sim$ 2\% и протонов с энергией около 1~ГэВ $\Delta p/p \sim$ 4\%.  

\subsection{ Магнит }

Магнитное поле создаётся при помощи сверхпроводящего магнита ISLE, который состоит из 6 секторов, которые в первом приближении отклоняют заряженные частицы в полярном направлении.
При максимальной силе тока $I$=3500~А максимум магнитного поля в 3 Тл достигается на краях магнита, а в центре сектора составляет 0.9 Тл.
Магнит фокусирует положительно заряженные частиц в направлении оси Z.
Сверхпроводящие катушки состоят из ниобий-титанового сплава, инкапсулированного в медную матрицу.
Медная матрица необходима для механической стабильности конструкции.
Вся сборка упакована в катушки, окруженные аллюминиевым корпусом, который предотвращает механические повреждения в случае внезапного отключения магнитного поля.
Катушки окружены системой охлаждения работающей на жидком азоте при температуре 85~К.
Токопроводящие эллементы дополнительно охлаждаются однофазным гелием при температуре 4.7~К при давлении 2.8 бар.

\subsection{ Камеры MDC }

Площадь чувсвительного материала внутренних камер составляет 0.35 м$^2$, а внешних --- 3.21 м$^2$.
Наименьшая чувствительная единица называется чувствительной ячейкой и представляет из себя плоскость с одним чувствительным и двумя потенциальными проволоками.
Катод и анод сделаны из отожженного аллюминия, а чувствительная проволока --- из покрытого золотом вольфрама.
Каждая секция состоит из порядка 1100 чувствительных ячеек, организованных в 6 слоёв, каждый из которых повёрнут на 20 градусов друг относительно друга ($\pm0^{\circ}$, $\pm20^{\circ}$, $\pm40^{\circ}$).
Такая организация чувствительного объема позволяет достичь равномерного разрешения по азимутальном (85-125 мкм) и полярному (35-50 мкм) углам.
Первый слой MDC заполнен смесью газов Ar+CO$_{2}$ в пропорциях 70:30.
Оставшиеся три слоя работают на смеси аргона и изобутана. 
Заряженная частица, пролетая через чувствительную зону детектора ионизирует газ, и высвобожденные электроны дрейфуют в сторону чувствительной проволоки.
Собранный заряд детектируется и восстанавливается пространственная координата в которой произошла ионизация газа.

\subsection{START и VETO детекторы}

START и VETO детекторы предназначены для детектирования событий столкновения ядер.
START и VETO используются для регистрации времени столкновения $T_0$ и выработки триггеров.
Совместно с времяпролётными детекторами TOF и RPC, START и VETO позволяют измерять время пролёта заряженных частиц.
Детектор VETO был разработан для подавления эффекта pile-up, когда на мишени происходит более одного взаимодействия.
Детектор имеет малую толщину, приблизительно 60 мкм, и состоит из алмазов, покрытых тонкой плёнкой металла. 
START детектор в свою очередь собран из алмазов с металлическим напылением, нанесенных тонким слоем на полоски из хромированного золота.
Всего 16 полосок шириной 200 мкм с интервалом в 90 мкм обеспечивают высокую точность регистрации налетающего ядра по $x$ и $y$.

\section{Времяпролётный детекторы TOF и RPC}

Времяпролётный детектор TOF состоит из 384 сцинтилярных стержней из поливинилтолуола, который обладает малой длиной ослабления света, высоким сцинтиляционным выходом и коротким временем распада.
Поперечные размеры внутренних стержней составляют $20 \times 20$ мм$^2$ и внешних --- $30\times30$ мм$^2$.
Проходя через сцинтиляционный стержень, заряженная частица возбуждает атомы чувствительного материала, которые затем возвращаются в основное состояние через эмиссию света.
Излученный свет распространяется в оба конца детектора, где считывается при помощи двух фотоумножителей.
По разности времён регистрации света на двух концах сцинтиляционного стержня затем рассчитывается $x$-координата попадания частицы.
Также по амплитуде сигнала рассчитываются энергопотери заряженной частицы при прохождении через материал детектора.

\section{}

\subsection{ Трекинговая система }

Трекинговая система HADES, предназначенная для реконструкции траекторий заряженных частиц, состоит из четырёх плоскостей многопроволочных дрейфовых камер (MDC).
Для измерения импульса заряженных частиц, между второй и третьей плоскостями, располагается сверхпроводящий магнит, отклоняющий проходящие через него частицы.
На рис.~\ref{fig:hades_tracking} схематически изображена трекинговая система эксперимента HADES.
Треки заряженных частиц совмещаются из траекторий в плоскостях I и II, и III и IV методом Рунге-Кутта.
Импульс заряженной частицы восстанавливается по отклонению в магнтином поле между плоскостями II и III 
%
\begin{figure}[ht]
\begin{center}
    \includegraphics[width=0.95\linewidth]{images/hades_tracking_system.png}
    \caption{Схмеатическое изображение трекинговой системы эксперимента HADES.}
    \label{fig:hades_tracking}
\end{center}
\end{figure}

Азимутальный аксептанс трекинговой системы в зависимости от быстроты частицы приведён на рис.~\ref{fig:hades_phi_y}.
Азимутальный аксептанс трекинговой системы не является однородным, поскольку стыки секций трекинговой системы не способны регистрировать заряженные частицы.
Неоднородность увеличивается с ростом быстроты, поскольку площадь нечувствительного объёма по отношению к чувствительному уменьшаяется с ростом полярного угла $\theta$. 
%
\begin{figure}[ht]
\begin{center}
\includegraphics[width=0.95\linewidth]{images/hades_phi_y.png}
\caption{Азимутальный аксептанс трекинговой системы эксперимента HADES в зависимости от быстроты частицы.}
\label{fig:hades_phi_y}
\end{center}
\end{figure}

\subsection{Эффективность реконструкции протонов}

Эффективность реконструкции протонов была рассчитана при помощи Монте-Карло моделирования отклика детектора. 
В качестве входных данных использовалась физическая модель DCM-QGSM-SMM.
Реалистичный отклик детекторов был смоделирован при помощи программного пакета GEANT3. 
Далее по модели отклика детектора была произведена реалистичная реконструкция.
Эффективность реконструкции определяется формулой:
\begin{equation}
    e(y, p_T) = \frac{ N_{rec}(y,p_T) }{ N_{sim}(y, p_T) },
\end{equation}
где $e(y, p_T)$ --- эффективность реконструкции для данных значений поперечного импульса ($p_T$) и быстроты ($y$), $N_{rec}$ --- число реконструированных частиц, $N_{sim}$ --- число смоделированных частиц.
На рис.~\ref{fig:hades_efficiency} представлена эффективность реконструкции протонов как функция быстроты ($y$) и поперечного импульса ($p_T$) для столкновений Au + Au при энергии $E_kin$=1.23$A$~ГэВ (слева), Ag + Ag при энергии $E_kin$=1.23$A$~ГэВ (посередине) и $E_kin$=1.58$A$~ГэВ (справа).
%
\begin{figure}[ht]
\begin{center}
\includegraphics[width=0.3\linewidth]{images/au123_efficiency_y_pT.png}
\includegraphics[width=0.3\linewidth]{images/ag123_efficiency_y_pT.png}
\includegraphics[width=0.3\linewidth]{images/ag158_efficiency_y_pT.png}
\caption{Эффективность реконструкции протонов как функция быстроты ($y$) и поперечного импульса ($p_T$) для столкновений Au + Au при энергии $E_kin$=1.23$A$~ГэВ (слева), Ag + Ag при энергии $E_kin$=1.23$A$~ГэВ (посередине) и $E_kin$=1.58$A$~ГэВ (справа). }
\label{fig:hades_efficiency}
\end{center}
\end{figure}

\subsection{Идентификация протонов времяпролётным методом}

Для измерения времени пролёта, установка HADES оборудована времяпролётными системами TOF и RPC, которые располагаются за трекинговой системой (см. рис.~\ref{fig:hades_bmn_layouts}).
Детектор TOF состоит из сцинтиляционных стержней, ориентированных радиально.
Детекторная подсистема RPC представляет из себя набор резистивных камер.
Идентификация частиц проводилась одновременно времяпролётным методом и по энерговыделению в камерах MDC.
На рис.~\ref{fig:hades_pid} представлено распределение заряженных частиц, зарегистрированных трекинговой системой HADES по относительной скорости $\beta$ и импульсу делённому на заряд $p/q$.
Используя соотношение:
\begin{equation}
    p = \frac{ m\beta }{ \sqrt{1-\beta^2} },
\end{equation}
где $p$ --- импульс частицы, $m$ --- ее масса, $\beta=v/c$, ее относительная скорость, можно рассчитать массу частицы.
%
\begin{figure}[ht]
    \begin{center}
    \includegraphics[width=0.95\linewidth]{images/hades_pid_plot.png}
    \caption{Распределение заряженных частиц, зарегистрированных трекинговой системой HADES по относительной скорости $\beta$ и импульсу делённому на заряд $p/q$.}
    \label{fig:hades_pid}
    \end{center}
    \end{figure}
    
Распределение заряженных частиц, зарегистрированных трекинговой системой HADES по квадрату массы $m^2$ и импульсу делённому на заряд $p/q$ представлено на рис.~\ref{fig:hades_m2_pq} (слева).
Ожидается, что массы рожденных частиц, измеренные времяпролётным методом будут распределены согласно нормальному распределению.
Среднее этого распределения для каждого типа частиц не должно зависить от импульса частицы, однако в эксперименте наблюдается сдвиг в сторону меньших значений для протонного пика.
Этот систмематический сдвиг может быть объяснён ошибкой при измерении частиц с малыми импульсами.
Большая кривизна траектории может приводить к ошибкам при ее реконструкции.
Ширина распределения для каждого вида частиц увеличивается с ростом импульса.
Этот эффект объясняется ограниченным разрешением времяпролётной системы, в которой при больших импульсах время пролёта восстанавливается с большей относительной ошибкой.
Каждый из пиков для разных типов частиц аппроксимируется функцией гаусса в узких диапазонах импульса.
Затем на основании этих аппроксимаций происходит отбор кандидатов в частицы для каждого типа.
Отобранные протоны представлены на рис.~\ref{fig:hades_m2_pq} (справа).
%
\begin{figure}[ht]
\begin{center}
\includegraphics[width=0.45\linewidth]{images/au123_m2_vs_pq_all.png}
\includegraphics[width=0.45\linewidth]{images/au123_m2_vs_pq_protons.png}
\caption{Распределение заряженных частиц, зарегистрированных трекинговой системой HADES по квадрату массы $m^2$ и импульсу делённому на заряд $p/q$: Для всех заряженных частиц (слева), для отобранных протонов (справа).}
\label{fig:hades_m2_pq}
\end{center}
\end{figure}

\subsection{Центральность столкновения}

Центральность столкновений в эксперименте HADES была определена на основе количества срабатываний времяпролетной системы.
Метод Монте-Карло Глаубера был использован для восстановления распределения геометрических параметров столкновения, таких как средний прицельный параметр, числа нуклонов-спектаторов и числа нуклонов-партисипантов (для деталей см.~\cite{HADES:2017def}).
На рис.~\ref{fig:hades_multiplicity} представлено Распределение множественности срабатываний времяпролетной системы в столкновениях Au + Au при энергии $E_{kin}$=1.23$A$~ГэВ и Ag + Ag при $E_{kin}$=1.23$A$~ГэВ и $E_{kin}$=1.58$A$~ГэВ.
Наибольшая множественность наблюдается в столкновениях Au + Au, поскольку число нуклов в ядрах золота почти в 2 раза больше.
\begin{figure}[ht]
\begin{center}
    \includegraphics[width=0.95\linewidth]{images/multiplicity.png}
    \caption{Распределение множественности заряженных срабатываний времяпролетной системы в столкновениях Au + Au при энергии $E_{kin}$=1.23$A$~ГэВ и Ag + Ag при $E_{kin}$=1.23$A$~ГэВ и $E_{kin}$=1.58$A$~ГэВ.
    }
    \label{fig:hades_multiplicity}
\end{center}
\end{figure}

Для каждой из анализируемых систем методом Монте-Карло Глаубера была измерена центральность столкновения. 
На рис.~\ref{fig:hades_centrality} представлено Распределение множественности срабатываний во времяпролётной системе в столкновениях Ag + Ag при энергии $E_{kin}$=1.58$A$~ГэВ. Вертикальными линиями обозначены границы классов центральности.
Аппроксимация множественности методом Монте-Карло Глаубера довольно хорошо описывает экспериментальное распределение в классе центральности 0-30\%.
Дальнейшее расхождение объясняется эффективностью центрального триггера.
Регистрация столкновения в эксперименте происходит по величине, пропорциональной множественности рожденных частиц.
Поэтому события с маленькой множественностью могут быть отброшены системой отбора событий столкновения.
Таким образом, экспериментальное распределение множественности частиц, рожденных в столкновении систематически смещено в область более высоких множественностей.
Монте-Карло розыгрыш множественности при помощи метода Глаубера и отрицательного биномиального распределения помогают оценить эту сдвижку и восстановить реальное распрделение множественности.  
%
\begin{figure}[ht]
\begin{center}
\includegraphics[width=0.75\linewidth]{images/hades_mult.png}
\caption{Распределение множественности заряженных срабатываний времяпролетной системы в столкновениях Ag + Ag при энергии $E_{kin}$=1.58$A$~ГэВ. Вертикальными линиями обозначены границы классов центральности.}
\label{fig:hades_centrality}
\end{center}
\end{figure}

\subsection{Передний годоскоп Forward Wall}

Для регистрации фрагментов сталкивающихся ядер, взаимодействовавших с областью перекрытия лишь упруго (спектаторы), спектрометр HADES оборудован годоскопом FW.
Детектор имеет модульную структуру и способен измерять заряд фрагментов-спектаторов.
Размер модулей годоскопа увеличивается от центральных к периферическим и составляет $40\times40$, $80\times80$ и $160\times160$~мм соответственно.
Схематично, расположение модулей в годоскопе представлено на рис.~\ref{fig:hodo_layout}.
%
\begin{figure}[ht]
\begin{center}
\includegraphics[width=0.75\linewidth]{images/FW_layout.jpg}
\caption{Схема расположения модулей переднего годоскопа FW.}
\label{fig:hodo_layout}
\end{center}
\end{figure}

На рис.~\ref{fig:fw_signal} представлено распределение сигнала в модулях сцинтиляционной стенки FW для столкновений Au + Au при $E_{kin}$=1.23$A$~ГэВ (слева), Ag + Ag при $E_{kin}$=1.23$A$~ГэВ (посередине) и $E_{kin}$=1.58$A$~ГэВ (справа).
Наиболее выраженный пик отвечает заряду $Z=1$ (charge=100).
Также наблюдаются пики для зарядов $Z=2$ и $Z=3$.
События срабатывания модулей стенки с большими зарядами фрагментов редки.
%
\begin{figure}[ht]
\begin{center}
\includegraphics[width=0.3\linewidth]{images/au123_fw_modules.png}
\includegraphics[width=0.3\linewidth]{images/ag123_fw_modules.png}
\includegraphics[width=0.3\linewidth]{images/ag158_fw_modules.png}
\caption{Распределение сигнала в модулях сцинтиляционной стенки FW для столкновений Au + Au при $E_{kin}$=1.23$A$~ГэВ (слева), Ag + Ag при $E_{kin}$=1.23$A$~ГэВ (посередине) и $E_{kin}$=1.58$A$~ГэВ (справа). }
\label{fig:fw_signal}
\end{center}
\end{figure}

В работе приводятся результаты анализа экспериментальных данных, полученные из столкновений ядер Au+Au при энергии $E_{kin}=$1.23$A$~ГэВ а также ядер Ag+Ag при энергиях $E_{kin}=$1.23$A$ и 1.58$A$~ГэВ, полученные на установке HADES.
Всего было проанализировано около 100 миллионов столкновений Au+Au и по 500 миллионов столкновений для Ag+Ag при обеих энергиях.
Для исследования использовались столкновения разделенные по времени и восстановленной вершиной лежащей в области мишени.
Траектории заряженных частиц были отобраны на основании качества аппроксимации трека.
Для отбора первичных частиц использовался критерий на минимальное расстояние между ее траекторией и первичной вершиной. 

\section{Описание экспериментальной установки BM@N}

\subsection{Схема установки}

Эксперимент Барионная Материя на Нуклотроне (BM@N) располагается на выделенном пучке ускорителя Nuclotron ускорительного комплекса NICA в ОИЯИ, Дубна.
BM@N является экспериментом с фиксированной мишенью.
Схема экспериментальной установке представлена на рис.~\ref{fig:bmn_layout}.
%
\begin{figure}[ht]
\begin{center}
\includegraphics[width=0.95\linewidth]{images/BM@N_layout.png}
\caption{Схема эксперимента BM@N.}
\label{fig:bmn_layout}
\end{center}
\end{figure}

\subsection{Трекинговая система}

Система реконструкции траекторий заряженных частиц в эксперименте BM@N состоит из четырех станций кремниевых детекторов (Silicon) и семи станций газо-электронных умножителей (GEM). 
В отличие от HADES, трекинговая система целиком находится в магнитном поле дипольного магнита что позволяет с большой точностью восстанавливать импульсы рожденных в столкновении заряженных частиц.
На рис.~\ref{fig:bmn_tracking} представлено схематическое изображение трекинговой системы в эксперименте BM@N.
Траектории заряженных частиц отклоняются магнитным полем дипольного магнита, что позволяет восстанавливать импульс заряженных частиц.
Вакуумная пучковая труба позволяет минимизировать столкновения ядер цезия с атомами азота, кислорода и прочими примесями.
Поскольку вакуумная труба также расположена в магнитном поле, она имеет искривлённую форму для свободного прохождения невзаимодействоваших ядер пучка.  
%
\begin{figure}[ht]
\begin{center}
    \includegraphics[width=0.75\linewidth]{images/bmn_tracking_system.png}
    \caption{Схематическое изображение трекинговой системы в эксперименте BM@N. Цифрами (1) обозначена мишень,
    (2) --- Barell Detector, (3) --- STS, (4) --- GEM, (5) --- Beam Pipe }
    \label{fig:bmn_tracking}
\end{center}
\end{figure}

В плоскости поперечной направлению пучка аксептанс установки имеет прямоугольную форму, что ведёт к значительной неоднородности аксептанса.
На рис.~\ref{fig:bmn_phi_eta} представлен азимутальный аксептанс заряженных частиц в зависимости от псевдобыстроты. 
%
\begin{figure}[ht]
\begin{center}
\includegraphics[width=0.75\linewidth]{images/bmn_phi_eta.png}
\caption{Азимутальный аксептанс заряженных частиц в зависимости от псевдобыстроты.}
\label{fig:bmn_phi_eta}
\end{center}
\end{figure}

На рис.~\ref{fig:bmn_mom_res} показано импульсное разрешение трекинговой системы как функция импульса частицы.
Различными цветами и маркерами показаны различные энергии столкновения ядер Xe и Cs.
Разрешение падает с уменьшением энергии столкновения. 
Этот эффект связан с уменьшением магнитного поля, при уменьшении энергии столкновения. 
При энергии $E_{kin}$=2$A$~ГэВ, экспериментальная установка работает с магнитным полем 0.4~Тл, при энергии $E_{kin}$=3$A$~ГэВ магнитное поле 0.6~Тл и при энергии $E_{kin}$=4$A$~ГэВ --- 0.8~Тл.
%
\begin{figure}[ht]
\begin{center}
\includegraphics[width=0.75\linewidth]{images/momentum_resolution.png}
\caption{Разрешение трекинговой системы по импульсу в эксперименте BM@N. Различными цветами и маркерами показана различная энергия столкновения.}
\label{fig:bmn_mom_res}
\end{center}
\end{figure}

\subsection{Определение центральности}

В эксперименте BM@N центральность также была определена при помощи метода Монте-Карло Глаубера, однако в качестве множественности использовалось число восстановленных траекторий заряженных частиц.
На рис.~\ref{fig:bmn_multiplicity} представлено распределение множественности заряженных частиц для Монте-Карло моделирования столкновений Xe+Cs(I) при энергии $E_{kin}$=4$A$~ГэВ.
Вертикальными линиями обозначены границы классов центральности.
Модель Монте-Карло Глаубера хорошо описывает распределение множественности в границах 0-60\% класса центральности.
%
\begin{figure}[ht]
\begin{center}
\includegraphics[width=0.75\linewidth]{images/bmn_multiplicity.png}
\caption{Распределение множественности заряженных частиц в эксперименте BM@N. Вертикальными линиями изображены границы классов центральности.}
\label{fig:bmn_multiplicity}
\end{center}
\end{figure}

\subsection{Времяпролётные детекторы TOF-400 и TOF-700}

В эксперименте BM@N идентификация заряженных частиц может выполняться только времяпролётным методом, используя информацию с двух станций времяпролётных детекторов, расположенных на расстоянии 400 и 700~см от мишении (TOF-400 и TOF-700 соответственно).
На рис.~\ref{fig:bmn_beta_pq} показано распределение заряженных частиц по относительной скорости $\beta=v/c$ и импульсу деленному на заряд $p/q$.
%
\begin{figure}[ht]
\begin{center}
\includegraphics[width=0.75\linewidth]{images/beta_pq.png}
\caption{Распределение заряженных частиц по относительной скорости $\beta=v/c$ и импульсу деленному на заряд $p/q$.}
\label{fig:bmn_beta_pq}
\end{center}
\end{figure}


Для каждого из детекторов была построена зависимость квадрата массы $m^2$ от импульса $p/q$. 
На рис.~\ref{fig:bmn_m2_pq} сверху, представлено распределение квадрата массы заряженной частицы в зависимости от импульса $p/q$ для TOF-400 (слева) и TOF-700 (справа).
В узких диапазонах поперечного импульса распределение частиц по квадрату массы было аппроксимировано гауссовой функцией. 
На рис.~\ref{fig:bmn_m2_pq} снизу, представлены кандидаты в протоны, которые лежат не дальше $2\sigma$ от пика квадрата массы.
%
\begin{figure}[ht]
\begin{center}
\includegraphics[width=0.95\linewidth]{images/bmn_m2_pq.png}
\caption{Распределение квадрата массы заряженной частицы в зависимости от импульса $p/q$ для TOF-400 (слева) и TOF-700 (справа). Сверху представлены распределения для всех заряженных частиц, снизу --- для отобранных протонов.}
\label{fig:bmn_m2_pq}
\end{center}
\end{figure}


На рис.~\ref{fig:bmn_pt_y} представлено распределение протонов по быстроте $y_{cm}$ и поперечному импульсу $p_T$ идентифицированных при помощи TOF-400 (слева сверху), TOF-700 (слева снизу), с использованием обоих TOF-детекторов (справа).
%
\begin{figure}[ht]
\begin{center}
\includegraphics[width=0.95\linewidth]{images/bmn_pt_y_acceptance.png}
\caption{Распределение протонов по быстроте $y_{cm}$ и поперечному импульсу $p_T$, идентифицированных при помощи TOF-400 (слева сверху), TOF-700 (слева снизу), с использованием обоих TOF-детекторов (справа).}
\label{fig:bmn_pt_y}
\end{center}
\end{figure}

\subsection{Передний адронный калориметр FHCal}

В эксперименте BM@N измеряется энерговыделение спектаторных фрагментов при помощи переднего адронного калориметра FHCal.
Адронный калориметр состоит из 54 модулей, их размеры --- $15\times15$ и $20\times20$~см соответственно.
Схема расположения модулей калориметра представлена на рис.~\ref{fig:fhcal_layout} справа.
Большие модули ($20\times20$~см) обозначены желтым цветом, малые модули ($15\times15$) обозначены синим цветом. 
%
\begin{figure}[ht]
\begin{center}
\includegraphics[width=0.95\linewidth]{images/FHCal_modules.png}
\caption{Схема расположения модулей переднего адронного калориметра FHCal.}
\label{fig:fhcal_layout}
\end{center}
\end{figure}

Разработанная физическая программа измерения коллективных потоков в эксперименте BM@N была проверена на реалистичном Монте-Карло моделировании отклика детектора, в качестве входных данных моделирования были использованы две физические модели столкновения тяжелых ионов.
Модель DCM-QGSM-SMM (Dubna Cascade Model, Quark-Gluon String Model, Statistical Multifragmentation Model) реалистично описывает выход спектаторных фрагментов, однако неудовлетворительно воспроизводит коллективную анизотропию рожденных частиц.
Эта модель была использована для проверки разработанных методов вычисления поправочного коэффициента разрешения в эксперименте BM@N.

Модель JAM (Jet-A-A Model) с импульсно-зависимым потенциалом дает реалистичный сигнал коллективной анизотропии рожденных барионов, однако в модели отсутствуют фрагменты с массовым номером $A>1$.
Данная модель была использована для проверки коррекций на неоднородность детектора и возможности алгоритмов реконструкции восстановить сигнал коллективной анизотропии.

\section{Выводы к главе 2}

В главе описывается устройство экспериментальной установки HADES. 
Рассмотрены принципы работы основных детекторных подсистем.
В главе описаны методы определения центральности и идентификации протонов с помощью экспериментальной установки HADES.
Представлены критерии отбора столкновений Au+Au и Ag+Ag а также заряженных частиц, рожденных в этих столкновениях.
Описаны способы вычисления эффективности реконструкции протонов при помощи программного пакета GEANT3.
В главе приведено краткое описание установки BM@N и ее детекторных подсистем.
Описываются методы измерения центральности столкновения по числу треков заряженных частиц. 
В главе обсуждаются способы измерения производительности установки BM@N для измерения направленного и эллиптического потоков протонов с помощью физических Монте-Карло моделей столкновений тяжелых ионов и программного пакета GEANT4.
