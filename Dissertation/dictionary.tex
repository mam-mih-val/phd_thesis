\chapter*{Словарь терминов}             % Заголовок
\addcontentsline{toc}{chapter}{Словарь терминов}  % Добавляем его в оглавление

\textbf{Фазовая диаграмма} --- графическое отображение равновесного состояния бесконечной физико-химической системы при условиях, отвечающих координатам рассматриваемой точки на диаграмме (носит название фигуративной точки).

\textbf{Бариохимический потенциал} ---  термодинамическая функция, применяемая при описании состояния систем с переменным числом частиц. 
Определяет изменение термодинамических потенциалов при изменении числа частиц в системе. 
Представляет собой адиабатическую энергию добавления одного бариона в систему без совершения работы.

\textbf{Прицельный параметр} --- отрезок, соединяющий центры сталкивающихся ядер.

\textbf{Плоскость реакции} --- плоскость определенная направлениями пучка и прицельного параметра.

\textbf{Плоскость симметрии} --- экспериментальная оценка плоскости реакции в конкретном событии столкновения ядер.

\textbf{Угол плоскости реакции (симметрии)} --- азимутальный угол плоскости реакции (симметрии).

\textbf{Нуклон-участник (партисипант)} --- нуклон, претерпевший неупругое рассеяние в процессе столкновения двух ядер.

\textbf{Нуклон-наблюдатель (спектатор)} --- нуклон, претерпевший упругое рассеяние в процессе столкновения двух ядер.

\textbf{Рожденная в столкновении частица} --- частица, которая является продуктом реакции неупругого расеяния нуклонов-участников.

\textbf{Поперечный импульс, $p_T$} --- проекция импульса на плоскость поперечную направлению пучка.

\textbf{Продольный импульс, $p_z$} --- проекция импульса на ось направления пучка.

\textbf{Полная энергия, $E$} --- релятивистская энергия частицы, $E = \sqrt{mc^2 + p^2}$.

\textbf{Быстрота} --- Величина, определенная по формуле $y = 0.5\ln \frac{ E + p_z }{ E - p_z }$, где $E$ --- полная энергия частицы, $p_z$ --- продольный импульс частицы.
Аддитивна относительно преобразований Лоренца.

\textbf{Коллективные анизотропные потоки} --- коэффициенты разложения в ряд Фурье азимутального распределения частиц относительно плоскости реакции.

\textbf{Направленный поток, $v_1$} --- первый коэффициент разложения в ряд Фурье азимутального распределения частиц относительно плоскости реакции.

\textbf{Эллиптический поток, $v_2$} --- второй коэффициент разложения в ряд Фурье азимутального распределения частиц относительно плоскости реакции.

\textbf{Центральность столкновения} --- отношение сечения взаимодействия данной группы столкновений к полному сечению неупругого взаимодействия.

\textbf{Непотоковые корреляции} --- корреляции, не связанные с коллективным движением частиц.

\textbf{Единичный вектор частицы, $u_n$} --- вектор, поставленный в соответствие каждой частицы в событии столкновения ядер. 
Определяется как $u_n = {\cos n\phi, \sin n\phi}$, где $\phi$ --- азимутальный угол частицы.

\textbf{Вектор события, $Q_n$} --- вектор, определенный как сумма по группе $u_n$-векторов в одном событии. 
Является оценкой ориентации плоскости реакции в данном событии. 

\textbf{Поправочный коэффициент разрешения плоскости симметрии, $R_n$} --- коэффициент, определенный как средний косинус разности угла плоскости реакции и плоскости симметрии $R_n = \langle \cos n (\Psi_S - \Psi_R) \rangle$, где $\Psi_R$ --- угол плоскости реакции, $\Psi_S$ --- угол плоскости симметрии. 

